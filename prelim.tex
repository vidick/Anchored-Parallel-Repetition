\section{Preliminaries}
\label{sec:prelim}

\subsection{Sets and indices} 
For an integer $n$, we let $[n] = \{1,\ldots,n\}$. For an alphabet $\X$, we let $\X^n$ denote the $n$-fold Cartesian product of $\X$. We denote elements of $\X^n$ by boldfaced letters $\bx = (\bx_1,\ldots,\bx_n)$. For a subset $C = \{i_1,\ldots,i_t\} \subseteq [n]$, we let $\bx_C$ denote the ordered tuple $(\bx_{i_1},\ldots,\bx_{i_t})$.

\subsection{Probability distributions, random variables, and expectations}\label{subsec:prob_dist}
We let capital letters denote random variables and lower case letters denote specific realizations of random variables. We use boldfaced variables to denote tuples, e.g., $\bX := (\bX_1,\ldots,\bX_n)$, $\bx = (\bx_1,\ldots,\bx_n)$. For a subset $C \subset [n]$ we write $\bX_C$ to denote the sub-tuple of $\bX$ indexed by $C$. We use $\P_X$ to denote the probability distribution of random variable $X$, and $\P_X(x)$ to denote the probability that $X = x$ for some value $x$. For multiple random variables, e.g., $X, Y, Z$, $\P_{XYZ}(x,y,z)$  denotes their joint distribution with respect to some probability space understood from context. 

We use $\P_{Y | X = x}(y)$ to denote the conditional distribution $\P_{YX}(y,x)/\P_X(x)$, which is defined when $\P_X(x) > 0$. When conditioning on many variables, we usually use the shorthand $\P_{X | y,z}$ to denote the distribution $\P_{X | Y =y,Z=z}$. For example, we write $\P_{V | \omega_\mi, \bx_i, \by_i}$ to denote $\P_{V | \Omega_\mi = \omega_\mi, \bX_i = \bx_i, \bY_i = \by_i}$. For an event $W$ we let $\P_{X Y | W}$ denote the distribution conditioned on $W$. We use the notation $\Ex_{X} f(x)$ and $\Ex_{\P_X} f(x)$ to denote the expectation $\sum_{x} \P_X(x) f(x)$. 

Let $\P_{X_0}$ be a distribution of $\X$, and for every $x$ in the support of $\P_{X_0}$, let $\P_{Y | X_1 = x}$ be a conditional distribution defined over $\Y$. We define the distribution $\P_{X_0} \P_{Y | X_1}$ over $\X \times \Y$ as
$$
	(\P_{X_0} \P_{Y | X_1})(x,y) \,:=\, \P_{X_0}(x) \cdot \P_{Y | X_1 = x}(y).
$$
Additionally, we write $\P_{X_0 Z} \P_{Y | X_1}$ to denote the distribution $(\P_{X_0 Z} \P_{Y | X_1})(x,z,y) := \P_{X_0 Z}(x,z) \cdot \P_{Y | X_1 = x}(y)$.

For two random variables $X_0$ and $X_1$ over the same set $\X$, $\P_{X_0} \approx_\eps \P_{X_1}$ indicates that the total variation distance between $\P_{X_0}$ and $\P_{X_1}$, 
$$\| \P_{X_0} - \P_{X_1} \| \,:=\, \frac{1}{2}\sum_{x \in \X} |\P_{X_0}(x) - \P_{X_1} (x)|,$$
 is at most $\eps$. 

We repeatedly use the following simple lemma throughout the paper:
\begin{lemma}\label{lem:trivial}
Let $\Qsf_F$ and $\Ssf_F$ be two probability distributions of some random variable $F$, and let $\Rsf_{G | F}$ be a conditional probability distribution for some random variable $G$, conditioned on $F$. Then
\[ \| \Qsf_{F} \Rsf_{G|F} - \Ssf_{F} \Rsf_{G|F}\|= \|\Qsf_{F}-\Ssf_{F}\|. \]
Similarly, if we have two conditional probability distributions $\Qsf_{G|F}, \Ssf_{G|F}$ and a distribution $\Rsf_F$, we have
\[
\| \Rsf_F \Qsf_{G|F} - \Rsf_F \Ssf_{G|F} \| = \Ex_F \| \Qsf_{G|F = f} - \Ssf_{G|F=f} \|
\]
where $\Ex_F$ denotes the expectation over sampling $f$ from $\Rsf_F$.
\end{lemma}
\begin{proof}
For the first part of the Lemma, we note that $ \| \Qsf_{F} \Rsf_{G|F} - \Ssf_{F} \Rsf_{G|F}\|$ is equal to
\begin{align*}
 \frac{1}{2}\sum_{f,g} | \Qsf(f) \Rsf(g|f)- \Ssf(f) \Rsf (g|f)|&=  \frac{1}{2}\sum_{f} |\Qsf(f)- \Ssf(f)| \cdot  \left( \sum_{g} \Rsf(g|f)  \right)\\
&= \frac{1}{2}\sum_{f} | \Qsf(f)- \Ssf(f)|\\
&= \|\Qsf_{F}-\Ssf_{F}\|.
\end{align*}
For the second part of the Lemma, we note that $\| \Rsf_F \Qsf_{G|F} - \Rsf_F \Ssf_{G|F} \|$ is equal to
\begin{align*}
\frac{1}{2}\sum_{f,g}	 | \Rsf(f) \Qsf(g|f) - \Rsf(f) \Ssf(g|f) | &= \sum_f \Rsf(f) \frac{1}{2} \sum_g	 | \Qsf(g|f) - \Ssf(g|f) | \\
&= \sum_f \Rsf(f) \| \Qsf_{G|F = f} - \Ssf_{G|F=f} \|\;.
\end{align*}
\end{proof}

\subsection{Quantum states and measurements}
All Hilbert spaces considered in this paper are finite-dimensional. We write $\Id$ to denote the identity operator. For Hermitian matrices $A, B$ we write $A \leq B$ to indicate that $A - B$ is positive semidefinite. For a linear operator $X$ acting on a complex vector space, we let $X[\cdot]$ to denote the map that takes linear operators $\rho \mapsto X\rho X^\dagger$.

A density matrix is a positive semidefinite matrix with trace $1$. A positive operator-valued measure (POVM) $M$ acting on $\C^d$ with outcome set $\A$ is denoted as a function $M$ mapping outcomes $a \in \A$ to positive semidefinite operators acting on $\C^d$ such that $\sum_a M(a) = \Id$. 

We use the convention that, when $\ket{\psi}$ is a pure state, $\psi$ refers to the rank-1 density matrix $\ketbra{\psi}{\psi}$. We use superscripts to denote system labels; so $\rho^{AB}$ denotes the density matrix on the systems $A$ and $B$. A \emph{classical-quantum (CQ)} state $\rho^{XE}$ is classical on $X$ and quantum on $E$ if it can be written as $\rho^{XE} = \sum_{x} p(x) \ketbra{x}{x}^X \otimes \rho^E_x$ for some probability measure $p(\cdot)$. The state $\rho^E_x$ defined to be the $E$ part of the state $\rho^{XE}$, conditioned on the classical register $X = x$. We write $\rho^{XE}_x$ to denote the state $\ketbra{x}{x}^X \otimes \rho^E_x$. %We often write expressions such as $\rho^{E|x}$ as shorthand for $\rho_{E|X = x}$ when it is clear from context which registers are being conditioned on. This will be useful when there are many classical variables to be conditioned on. 

We generally decorate states (both pure states and density matrices) with superscripts to indicate the spaces and registers they reside in. 


For a vector $\ket{\psi}$, we use $\| \ket{\psi} \|$ to denote its Euclidean length. For a matrix $A$, we use $\| A \|_1$ to denote its \emph{trace norm} $\Tr(\sqrt{A A^\dagger})$. 

%\subsection{Norms and distance measures}



\subsection{Nonlocal games}

\begin{definition}[Nonlocal game]
A \emph{nonlocal game} $G$ (or \emph{game} for short) is a tuple $(\X \times \Y,\A \times \B,\mu,V)$ where $\X, \Y$ are finite \emph{question sets}, $\A, \B$ are finite \emph{answer sets}, $\mu$ is a \emph{question distribution} over $\X \times \Y$, and $V: \X \times \Y \times \A \times \B \to \{0,1\}$ is a \emph{game predicate}.
\end{definition}

\begin{definition}[Parallel repetition of a nonlocal game]
Let $G = (\X \times \Y,\A \times \B,\mu,V)$ be a nonlocal game. Then its \emph{$n$-fold parallel repetition} is a nonlocal game $G^n = (\X^n \times \Y^n, \A^n \times \B^n,\mu^n,V^n)$ where $\mu^n$ is the product distribution $\mu \times \cdots \times \mu$ over $(\X \times \Y)^n$, and $V^n: \X^n \times \Y^n \times \A^n \times \B^n \to \{0,1\}$ is the game predicate defined as
\[
	V^n(\bx,\by,\ba,\bb) = \prod_{i=1}^n V(\bx_i,\by_i,\ba_i,\bb_i)
\]
for all $\bx \in \X^n,\by \in \Y^n,\ba \in \A^n,\bb \in \B^n$.
\end{definition}

\begin{definition}[Quantum strategies]
Let $G = (\X \times \Y,\A \times \B,\mu,V)$ be a nonlocal game. A \emph{quantum strategy $\strategy$ for the game $G$} is a tuple $(\ket{\psi}, A,B)$ where $\ket{\psi}$ is a bipartite state in $\C^d \times \C^d$ for some positive integer $d$, and $A = \{ A_x \}_{x \in \X}$ and $B = \{B_y\}_{y \in \Y}$ are collections of POVMs (with outcomes in $\A$ and $\B$, respectively) acting on $\C^d$. The \emph{dimension of $\strategy$} is $d$. The \emph{quantum value of the strategy $\strategy$ in the game $G$} is defined as
\[
	\eval(G,\strategy) = \sum_{(x,y) \in \X \times \Y} \mu(x,y) \,  \sum_{(a,b) \in \A \times \B} V(x,y,a,b) \, \bra{\psi} A_x(a) \otimes B_y(b) \ket{\psi}.
\]
\end{definition}

\begin{definition}[Quantum value of a nonlocal game]
Let $G = (\X \times \Y,\A \times \B,\mu,V)$ be a nonlocal game. The \emph{quantum value of $G$} is defined as
\[
	\eval(G) = \sup_{\strategy} \eval(G,\strategy)
\]
where the supremum is over the set of all quantum strategies for $G$.
\end{definition}

\begin{definition}[Entanglement requirements of a nonlocal game]
Let $G = (\X \times \Y,\A \times \B,\mu,V)$ be a nonlocal game, and let $p \in [0,1]$. Define $\E(G,p)$ to denote the minimum integer $d \in \N$ such that there exists a $d$-dimensional quantum strategy $\strategy$ for $G$ such that $\eval(G,\strategy) \geq p$. If there is no such strategy, then we define $\E(G,p) = +\infty$.
\end{definition}

%\begin{proposition}[Entanglement lower bound implies value upper bound]
%\label{prop:entanglement-lb-to-value}
%If $\E(G,p) = +\infty$, then $\eval(G) \leq p$.
%\end{proposition}
%\begin{proof}
%Suppose $\eval(G) > p$. Then there exists a $d$-dimensional strategy $\strategy$ for $G$ for some $d \in \N$ that achieves value $p$, which implies that $\E(G,p) \leq d$, a contradiction.
%\end{proof}

